\documentclass{scrartcl}

\title{Some notes on the CIECAM02 and CAM16 color models}
\author{Nico Schlömer}

\begin{document}

\maketitle
\begin{abstract}
  The CIECAM02 color model \cite{} was revised in 2016 \cite{}, mainly to
  eliminate one particular mathematical breakdown. Some little inconsistencies
  however have survived the update. This note points them out in the hope that
  the next revision will be cleaned up in this respect.
\end{abstract}


\section{Description of the algorithm}

For the sake of completeness, the section repeats the CAM16 color model as it
originally appears in \cite{}.

\subsection{Forward model}

\subsection{backward model}


\section{Possible improvements}

All improvements listed here do not change the outcome of the model.  They are
merely algorithmic improvements that leads to simplified implementations.

\begin{itemize}

\item Step 3 in the forward model can take profit from the sign definition of
step 5 of the backward model  One expression instead of two.

\item Step 5 in the forward model is written down too complicated. Use the same
grammar as in step 1-3 of the backward model

\item
In both CAM02 and CAM16, the inverse model's step 3 is unecessarily complex.
Instead of discriminating between $t=0$ and $t>0$ to avoid a division by $0$
in $p_1$, one can simply perform all computations with the inverse of $p_1$,
\[
\tilde{p}_1 = t e_t \frac{13}{50000} N_c N_{cb}.
\]
This leads to a simpler representation of $a$ and $b$.
\end{itemize}


\section{}
On Encoding Color Difference Signals for High Dynamic Range and Wide Gamut
Imagery and derivative articles like Perceptually uniform color space for image
signals including high dynamic range and wide gamut:

It says there about matrix $M_1$:

> For matrix M1 the signal
> range for !, ! and ! should remain equal to the range of the input
> RGB because the nonlinear conversion function is only defined for
> zero to ten thousand. Thus the sum of the coefficients in each row
> of matrix !! must equal one.

Their final matrix is

0.37613 0.70431 −0.05675
−0.21649 1.14744 0.05356
0.02567 0.16713 0.74235

However, to assert that the output signal is also in the range of 0 and 10000,
there can be no negative entries. This requirement has been forgotten.

Proof:
If any $M_{i_0, j_0}<0$, then chose
$x$ with $x_j={1 if j=j_0, 0 otherwise}$. Then $(Mx)_{i_0} = M_{i_0, j_0} x_j
=M_{i_0, j_0} < 0$. Hence, there cannot be negative entries.

To maximize entry in $(Mx)_i$, one must choose $x=x_{max} (1,1,1)$. Then
$(Mx)_i = \sum_j M_{i, j} x_{max} \le! x_{max}$, hence $\sum_j M_{i, j} \le!
1$, i.e., the sum of each row of $M$ can at most be $1$.

\end{document}
