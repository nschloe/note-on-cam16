\documentclass[twocolumn]{scrartcl}

% Step environment
% <https://tex.stackexchange.com/a/12943/13262>
\usepackage{amsthm}
\newtheorem*{theorem}{Step}
\newtheoremstyle{named}{}{}{\itshape}{}{\bfseries}{.}{.5em}{\thmnote{#1 }#3}
\theoremstyle{named}
\newtheorem*{step}{Step}

\usepackage{microtype}
\usepackage{amsmath}
\usepackage{mathtools}
\usepackage{booktabs}
\usepackage{tabularx}

% <https://tex.stackexchange.com/a/43009/13262>
\DeclarePairedDelimiter\abs{\lvert}{\rvert}%

\usepackage[T1]{fontenc}
\usepackage{newtxtext}
\usepackage{newtxmath}

% degree symbol
\usepackage{gensymb}

% <https://tex.stackexchange.com/a/413899/13262>
\usepackage{etoolbox}
\makeatletter
\long\def\etb@listitem#1#2{%
  \expandafter\ifblank\expandafter{\@gobble#2}
    {}
    {\expandafter\etb@listitem@i
     \expandafter{\@secondoftwo#2}{#1}}}
\long\def\etb@listitem@i#1#2{#2{#1}}
\makeatother
\usepackage{biblatex}
\bibliography{bib}

\usepackage{amsmath}
\DeclareMathOperator{\sign}{sign}

\usepackage{bm}
\newcommand\rgb{\bm{R}}

\title{Some notes on the CIECAM02 and CAM16 color models}
\author{Nico Schlömer}

\begin{document}

\maketitle
\begin{abstract}
  The CIECAM02 color appearance model~\cite{ciecam02} has attracted much
  attention, but it was quickly discovered that it breaks down for certain
  input values. Some articles have since been published trying to fix the
  breakdown, most recently 2016~\cite{cam16}. The algorithm still contains some
  flaws, however. This article suggests some improvement that lead that do not
  change the algorithm algebraically, but make it shorter and remove some edge
  cases in which the current algorithm would still break down.
\end{abstract}


% \section{Description of the algorithm}
%
% For the sake of completeness, the section repeats the CAM16 color model as it
% originally appears in \cite{}.
%
% \subsection{Forward model}
%
% \subsection{backward model}

\section{Suggested improvements}

All improvements listed here do not change the outcome of the model.  They are
merely algorithmic improvements that leads to simplified implementations.

\subsection{Step 3, forward model}

Step 3 of the forward model reads

\begin{step}[3]
Calculate the postadaptation cone response
(resulting in dynamic range compression).
\[
  R_a = 400 \left(\frac{\left(\frac{F_L R_c}{100}\right)^{0.42}}{\left(\frac{F_L R_c}{100}\right)^{0.42} + 27.13}\right) + 0.1
\]
If $R_c$ is negative, then
\[
  R_a = -400 \left(\frac{\left(\frac{-F_L R_c}{100}\right)^{0.42}}{\left(\frac{-F_L R_c}{100}\right)^{0.42} + 27.13}\right) + 0.1
\]
and similarly for the computations of $G_a$ and $B_a$.
\end{step}

If the $\sign$ operator is used here as it is used later in step 5 of the
backward model, the above description can be shortened.
Furthermore, the term $0.1$ is added here, but in all of the following steps in
which $\rgb_a$ is used -- except the computation of $t$ in step 9 --, it
cancels out or is subtracted again. It is hence suggested to include the term
only in step 9.

\subsection{Step 9, forward model}

The saturation is currently computed as
\begin{step}[9]
  Calculate the correlates of [\dots] saturation ($s$).
  \[
    s \coloneqq 100 \sqrt{M/Q}.
  \]
\end{step}
This breaks down if $Q=0$, a value only occuring if the input values are
$X=Y=Z=0$. When making use of the definition of $M$ and $Q$, one gets to an
expression for $s$ that is well-defined in all cases:
\[
  s \coloneqq 50 \sqrt{\frac{c t^{0.9} {(1.64-0.29)}^{0.73}}{A_w + 4}}.
\]


\subsection{Steps 2 and 3, backward model}

\begin{step}[2]
Calculate $t$, $e_t$, $p_1$, $p_2$, and $p_3$.
\begin{align*}
  t &= {\left(\frac{C}{\sqrt{\frac{J}{100}} {(1.64 - 0.29^n)}^{0.73}}\right)}^\frac{1}{0.9},\\
  e_t &= \frac{1}{4} \left[\cos(h'\pi/180\degree + 2) + 3.8\right],\\
  p_1 &= \frac{50000}{13} N_c N_{cb} e_t \frac{1}{t},\\
  p_2 &= \frac{A}{N_{bb}} + 0.305,\\
  p_3 &= \frac{21}{20}.
\end{align*}
\end{step}

\begin{step}[3]
Calculate $a$ and $b$.
If $t=0$, then $a=b=0$ and go to Step 4.
In next computations be transform $h$ from degrees to radians
before calculating $\sin(h)$ and $\cos(h)$:
If $\abs{\sin(h)} \ge \abs{\cos(h)}$ then
\begin{align*}
  p_4 &= \frac{p_1}{\sin(h)},\\
  b &= \frac{p_2 (2+p_3) \frac{460}{1403}}{p_4 + (2+p_3) \frac{220}{1403} \frac{\cos(h)}{\sin(h)} - \frac{27}{1403} + p_3 \frac{6300}{1403}},\\
  a &= b \frac{\cos(h)}{\sin(h)}.
\end{align*}
If $\abs{\cos(h)} > \abs{\sin(h)}$ then
\begin{align*}
  p_5 &= \frac{p_1}{\cos(h)},\\
  a &= \frac{p_2 (2+p_3) \frac{460}{1403}}{%
    p_5
    + (2+p_3) \frac{220}{1403} -
    \left(\frac{27}{1403}  - p_3 \frac{6300}{1403}\right) \frac{\sin(h)}{\cos(h)}
  },\\
  b &= a \frac{\sin(h)}{\cos(h)}.
\end{align*}
\end{step}

Some of the complications in this step stem from the fact that the variable $t$
might be $0$ in the definition of $p_1$. This happens if and only if the input
parameter $C$ is $0$.
One can instead perform the computation without explicitly dividing by $t$;
namely, in the case $\abs{\sin(h)} \ge \abs{\cos(h)}$:
\begin{align*}
  p'_1 &\coloneqq \frac{50000}{13} N_c N_{cb} e_t,\\
  b &= \frac{p_2 (2+p_3) \frac{460}{1403}}{\frac{p'_1}{t\sin(h)} + (2+p_3) \frac{220}{1403} \frac{\cos(h)}{\sin(h)} - \frac{27}{1403} + p_3 \frac{6300}{1403}}\\
   &= \frac{t \sin(h) p_2 (2+p_3) \frac{460}{1403}}{p'_1 + t (2+p_3) \frac{220}{1403} \cos(h) + t \sin(h) \frac{6588}{1403}}\\
   &= \frac{23 t \sin(h) p_2}{23 p'_1 + 11 t \cos(h) + 108 t \sin(h)},
\end{align*}
and
\[
  a = \frac{23 t \cos(h) p_2}{23 p'_1 + 11 t \cos(h) + 108 t \sin(h)}.
\]
The exact same expressions are retrieved in the case
$\abs{\cos(h)} > \abs{\sin(h)}$.
It is also clear that they are always well-defined when realizing that
\begin{multline*}
  23 p'_1 + 11 t \cos(h) + 108 t \sin(h)\\
  = \frac{23 p'_1 p_2}{R'_a + G'_a + \tfrac{21}{20}B'_a + 0.305}
  \neq 0.
\end{multline*}

The value of $t$ can be retrieved via $\alpha$ from the input variables.
If the saturation correlate $s$ is given, one has
\[
  \alpha \coloneqq {\left(\frac{s}{50}\right)}^2 \frac{A_w+4}{c};
\]
if $M$ is given, compute $C\coloneqq M / F_L^{0.25}$ and then
\[
\alpha\coloneqq\begin{dcases*}
  0 &if $J=0$,\\
  \frac{C}{\sqrt{J/100}}&otherwise.
\end{dcases*}\]



\section{Full model}

All steps that differ from the original model are marked with an asterisk (*).

As an abbreviation, the bold $\rgb$ is used whenever the equation applied
to $R$, $G$, and $B$ alike.

\subsection{Forward model}

\begin{step}[0*]
Calculate all values/parameters which are independent
of the input sample.
\begin{align*}
  &\begin{pmatrix}R_w\\G_w\\B_w\end{pmatrix}
    = M_{16}
  \begin{pmatrix}X_w\\Y_w\\Z_w\end{pmatrix},\\
  &D = F \left[1 - \tfrac{1}{3.6} \exp\left(\tfrac{-L_a-42}{92}\right)\right].
\end{align*}
If $D$ is greater than one or less than zero, set it to one or zero,
respectively.
\begin{align*}
  &D_{\rgb} = D\frac{Y_W}{\rgb_W} -1 + D,\\
  &k = \frac{1}{5L_A + 1},\\
  &F_L = k^4 L_A + 0.1 {(1-k^4)}^2 {(5L_A)}^{1/3},\\
  &n = \frac{Y_b}{Y_W},\\
  &z = 1.58 + \sqrt{n},\\
  &N_{bb} = \frac{0.725}{n^{0.2}},\\
  &N_{cb} = N_{bb},\\
  &\rgb_{wc} = D_{\rgb} \rgb_w,\\
  &\rgb_{aw} = 400
  \frac
  {{\left(\frac{F_L \rgb_{wc}}{100}\right)}^{0.42}}
  {{\left(\frac{F_L \rgb_{wc}}{100}\right)}^{0.42} + 27.13},\\
  &A_w = \left(2R_{aw} + G_{aw} + \tfrac{1}{20} B_{aw}\right) \cdot N_{bb}.
\end{align*}
\end{step}

\begin{table}\centering
  \begin{tabularx}{\linewidth}{XXXX}
  \toprule
          & $F$ & $c$   & $N_c$\\
  \midrule
  Average & 1.0 & 0.69  & 1.0\\
  Dim     & 0.9 & 0.59  & 0.9\\
  Dark    & 0.8 & 0.525 & 0.8\\
  \bottomrule
\end{tabularx}
\caption{Surround parameters.}
\end{table}


\begin{table}\centering
  \begin{tabularx}{\linewidth}{XXXXXX}
  \toprule
        & Red   & Yellow & Green & Blue   & Red\\
  \midrule
  $i$   & 1     & 2     & 3      & 4      & 5\\
  $h_i$ & 20.14 & 90.00 & 164.25 & 237.53 & 380.14\\
  $e_i$ & 0.8   & 0.7   & 1.0    & 1.2    & 0.8\\
  $H_i$ & 0.0   & 100.0 & 200.0  & 300.0  & 400.0\\
  \bottomrule
\end{tabularx}
  \caption{Unique hue data for calculation of hue quadrature.}\label{table:hue}
\end{table}

\begin{step}[1]
Calculate `cone' responses.
\[
\begin{pmatrix}R\\G\\B\end{pmatrix}
= M_{16} \begin{pmatrix}X\\Y\\Z\end{pmatrix}
\]
\end{step}

\begin{step}[2]
Complete the color adaptation of the illuminant in
the corresponding cone response space (considering various
luminance levels and surround conditions included in $D$, and
hence in $D_R$, $D_G$, and $D_B$).
\[
  \rgb_c = D_{\rgb} \cdot \rgb
\]
\end{step}

\begin{step}[3*]
Calculate the modified postadaptation cone response
(resulting in dynamic range compression).
\[
  \rgb'_a = 400 \sign(\rgb_c)
    \frac
    {{\left(\frac{F_L \abs{\rgb_c}}{100}\right)}^{0.42}}
    {{\left(\frac{F_L \abs{\rgb_c}}{100}\right)}^{0.42} + 27.13}.
\]
\end{step}

\begin{step}[4]
Calculate Redness--Greenness ($a$), Yellowness--Blueness ($b$) components,
and hue angle ($h$):
\begin{align*}
  a&\coloneqq R'_a - \tfrac{12}{11} G'_a + \tfrac{1}{11} B'_a\\
  b&\coloneqq \tfrac{1}{9} R'_a + \tfrac{1}{9} G'_a - \tfrac{2}{9} B'_a\\
  h&\coloneqq \arctan(b/a).
\end{align*}
(Make sure that $h$ is between $0\degree$ and $360\degree$.)
\end{step}

\begin{step}[5]
Calculate eccentricity [$e_t$, hue quadrature composition
($H$) and hue composition ($H_c$)].

Using the following unique hue data in table~\ref{table:hue}, set
$h'= h + 360\degree$ if $h < h_1$, otherwise $h'=h$.
Choose a proper $i\in\{1,2,3,4\}$ so that $h_i\le h' < h_{i+1}$.
Calculate
\[
  e_t = \tfrac{1}{4}
  \left[
    \cos(h'\pi/180\degree + 2) + 3.8
  \right]
\]
which is close to, but not exactly the same as, the eccentricity factor given
in table~\ref{table:hue}.

Hue quadrature is computed using the formula
\[
  H = H_i + \frac{100 e_{i+1} (h'-h_i)}{e_{i+1}(h'-h_i) + e_i (h_{i+1}-h')}
\]
and hue composition $H_c$ is computed according to $H$.  If $i=3$ and $H =
241.2116$ for example, then $H$ is between $H_3$ and $H_4$ (see
table~\ref{table:hue} above). Compute $P_L=H_4-H = 58.7884$; $P_R = H – H_3 =
41.2116$ and round $P_L$ and $P_R$ values to integers $59$ and $41$. Thus,
according to table~\ref{table:hue}, this sample is considered as having 59\%
of green and 41\% of blue, which is the $H$c and can be reported as 59G41B or
41B59G.
\end{step}

\begin{step}[6*]
Calculate achromatic response $A$
\[
  A\coloneqq (2 R'_a + G'_a + \tfrac{1}{20} B'_a) \cdot N_{bb}
  \]
\end{step}

\begin{step}[7]
Calculate the correlate of lightness $J$
\[
  J \coloneqq 100 {(A / A_w)}^{cz}
  \]
\end{step}

\begin{step}[8]
  Calculate the correlate of brightness $Q$
  \[
    Q \coloneqq \frac{4}{c} \sqrt{\frac{J}{100}} (A_w+4) F_L^{0.25}
    \]
\end{step}

\begin{step}[9*]
Calculate the correlates of chroma ($C$), colorfulness ($M$), and saturation
  ($s$).
\begin{align*}
  t&\coloneqq \frac{50000/13 N_c N_{cb} e_t \sqrt{a^2 + b^2}}{R'_a + G'_a + 21/20 B'_a + 0.305},\\
  \alpha&\coloneqq t^{0.9} \cdot {(1.64 - 0.29^n)}^{0.73},\\
  C&\coloneqq \alpha \sqrt{\frac{J}{100}},\\
  M&\coloneqq C\cdot F_L^{0.25},\\
  s &\coloneqq 50 \sqrt{\frac{\alpha c}{A_w + 4}}.
\end{align*}
\end{step}

\subsection{Backward model}

\begin{step}[1]
  Obtain $J$, $C$; and $h$ from $H$, $Q$, $M$, $s$.

  The input data can be different combinations of perceived correlates, that
  is, $J$ or $Q$; $C$, $M$, or $s$; and $H$ or $h$. Hence the following
  sub-steps are needed to convert the input parameters to the parameters $J$,
  $C$, and $h$.
\end{step}

\begin{step}[1--1]
Compute $J$ from $Q$ (if input is $Q$)
\[
  J\coloneqq 6.25 \left(\frac{cQ}{(A_w+4) F_L^{0.25}}\right).
\]
\end{step}

\begin{step}[1--2*]
Calculate $t$ from $C$, $M$, or $s$.
\begin{itemize}
  \item If input is $C$ or $M$:
    \begin{align*}
      C &\coloneqq M / F_L^{0.25} \:\text{if input is $M$}\\
      \alpha &\coloneqq \begin{dcases*}
          0 &if $J=0$,\\
          \frac{C}{\sqrt{J/100}}& otherwise.
      \end{dcases*}
    \end{align*}
  \item If input is $s$:
    \[
    \alpha \coloneqq {\left(\frac{s}{50}\right)}^2 \frac{A_w+4}{c}
    \]
\end{itemize}
Compute $t$ from $\alpha$:
\[
  t \coloneqq {\left(\frac{\alpha}{{(1.64 - 0.29^n)}^{0.73}}\right)}^{1/0.9}
\]
\end{step}

\begin{step}[1--3]
Calculate $h$ from $H$ (if input is $H$).
The correlate of hue ($h$) can be computed by using data in
table~\ref{table:hue} in the forward model.
Choose a proper $i\in\{1,2,3,4\}$ such that
$H_i \le H < H_{i+1}$. Then
\[
  h' = \frac{(H-H_i)(e_{i+1}h_i - e_i h_{i+1}) - 100 h_i e_{i+1}}{(H-H_i)(e_{i+1}-e_i) - 100 e_{i+1}}.
\]
Set $h = h' - 360\degree$ if $h' > 360\degree$, and $h=h'$ otherwise.
\end{step}

\begin{step}[2*]
Calculate $e_t$, $A$, $p'_1$, and $p'_2$
\begin{align*}
  e_t &= \tfrac{1}{4} (\cos(h\pi/180\degree + 2) + 3.8),\\
  A &= A_w  {(J/100)}^{1/(cz)},\\
  p'_1 &= e_t \tfrac{50000}{13} N_c N_{cb},\\
  p'_2 &= A / N_{bb}.
\end{align*}
\end{step}

\begin{step}[3*]
Calculate $a$ and $b$
  \begin{align*}
    \gamma &\coloneqq \frac{23 (p'_2+0.305) t}{23 p'_1 + 11 t \cos(h) + 108 t \sin(h)},\\
    a &\coloneqq \gamma \cos(h),\\
    b &\coloneqq \gamma \sin(h).
  \end{align*}
\end{step}

\begin{step}[4]
  Calculate $R'_a$, $G'_a$, and $B'_a$.
  \begin{align*}
    R'_a &= (460 p'_2 + 451 a + 288 b) / 1403,\\
    G'_a &= (460 p'_2 - 891 a - 261 b) / 1403,\\
    B'_a &= (460 p'_2 - 220 a - 6300 b) / 1403.
  \end{align*}%
\end{step}

\begin{step}[5*]
Calculate $R_c$, $G_c$, and $B_c$,
  \[
  \rgb_c = \sign(\rgb'_a)
  \frac{100}{F_L} {\left(
    \frac{27.13 \abs{\rgb'_a}}{400 - \abs{\rgb'_a}}
    \right)}^{1/0.42}.
  \]
\end{step}

\begin{step}[6]
Calculate $R$, $G$, and $B$ from $R_c$, $G_c$, and $B_c$.
\[
  \rgb = \rgb_c / D_{\rgb}.
\]
\end{step}

\begin{step}[7]
Calculate $X$, $Y$, and $Z$ (for the coefficients of the inverse matrix, see
the note at the end of the appendix B).  % TODO
\[
\begin{pmatrix}X\\Y\\Z\end{pmatrix}
  = M_{16}^{-1}
\begin{pmatrix}R\\G\\B\end{pmatrix}.
\]
\end{step}



% \section{}
% On Encoding Color Difference Signals for High Dynamic Range and Wide Gamut
% Imagery and derivative articles like Perceptually uniform color space for image
% signals including high dynamic range and wide gamut:
%
% It says there about matrix $M_1$:
%
% > For matrix M1 the signal
% > range for !, ! and ! should remain equal to the range of the input
% > RGB because the nonlinear conversion function is only defined for
% > zero to ten thousand. Thus the sum of the coefficients in each row
% > of matrix !! must equal one.
%
% Their final matrix is
%
% 0.37613 0.70431 −0.05675
% −0.21649 1.14744 0.05356
% 0.02567 0.16713 0.74235
%
% However, to assert that the output signal is also in the range of 0 and 10000,
% there can be no negative entries. This requirement has been forgotten.
%
% Proof:
% If any $M_{i_0, j_0}<0$, then chose
% $x$ with $x_j={1 if j=j_0, 0 otherwise}$. Then $(Mx)_{i_0} = M_{i_0, j_0} x_j
% =M_{i_0, j_0} < 0$. Hence, there cannot be negative entries.
%
% To maximize entry in $(Mx)_i$, one must choose $x=x_{max} (1,1,1)$. Then
% $(Mx)_i = \sum_j M_{i, j} x_{max} \le! x_{max}$, hence $\sum_j M_{i, j} \le!
% 1$, i.e., the sum of each row of $M$ can at most be $1$.

\printbibliography{}

\end{document}
